\documentclass[conference]{IEEEtran}
\usepackage{graphicx}
\usepackage{amsmath,amssymb,amsfonts}
\usepackage{textcomp}
\usepackage{xcolor}

\begin{document}

\title{Análisis de Rendimiento de Dotplot Secuencial vs. Paralelización}

\author{
\IEEEauthorblockN{Juan José González}
\IEEEauthorblockA{\textit{Institución} \\
Dirección \\
Correo electrónico}
}

\maketitle

\begin{abstract}
En este trabajo, se realiza un análisis comparativo de las técnicas de dotplot para la comparación de secuencias genéticas ejecutadas de forma secuencial y paralela. Se utilizan tres técnicas diferentes: secuencial, multiprocessing y MPI, y se analiza su rendimiento en varios aspectos.
\end{abstract}

\section{Introducción}
Explicación del problema del dotplot y su relevancia en bioinformática. Discusión de la necesidad de técnicas de paralelización para manejar grandes volúmenes de datos.

\section{Implementación}
Descripción detallada de las tres implementaciones: secuencial, con multiprocessing y con MPI.

\section{Metodología de análisis de rendimiento}
Descripción de cómo se calculan las métricas de rendimiento y qué representan.

\section{Resultados}
Presentación de los resultados del análisis de rendimiento. Incluir gráficos para ilustrar las diferencias en el rendimiento entre las tres implementaciones.

\subsection{Secuencial}
Descripción de los resultados de la implementación secuencial.

\subsection{Multiprocessing}
Descripción de los resultados de la implementación con multiprocessing.

\subsection{MPI}
Descripción de los resultados de la implementación con MPI.

\section{Discusión}
Análisis de los resultados y comparación entre las tres implementaciones. Discusión de las ventajas y desventajas de cada implementación.

\section{Conclusiones}
Resumen de los hallazgos principales del proyecto.

\section*{Agradecimientos}
Agradecimientos a quienes te han ayudado en el proyecto.

\begin{thebibliography}{00}
\bibitem{b1} Referencia 1
\bibitem{b2} Referencia 2
\end{thebibliography}

\end{document}
